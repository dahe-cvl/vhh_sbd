%% Generated by Sphinx.
\def\sphinxdocclass{report}
\documentclass[letterpaper,10pt,english,openany,oneside]{sphinxmanual}
\ifdefined\pdfpxdimen
   \let\sphinxpxdimen\pdfpxdimen\else\newdimen\sphinxpxdimen
\fi \sphinxpxdimen=.75bp\relax

\PassOptionsToPackage{warn}{textcomp}
\usepackage[utf8]{inputenc}
\ifdefined\DeclareUnicodeCharacter
% support both utf8 and utf8x syntaxes
  \ifdefined\DeclareUnicodeCharacterAsOptional
    \def\sphinxDUC#1{\DeclareUnicodeCharacter{"#1}}
  \else
    \let\sphinxDUC\DeclareUnicodeCharacter
  \fi
  \sphinxDUC{00A0}{\nobreakspace}
  \sphinxDUC{2500}{\sphinxunichar{2500}}
  \sphinxDUC{2502}{\sphinxunichar{2502}}
  \sphinxDUC{2514}{\sphinxunichar{2514}}
  \sphinxDUC{251C}{\sphinxunichar{251C}}
  \sphinxDUC{2572}{\textbackslash}
\fi
\usepackage{cmap}
\usepackage[T1]{fontenc}
\usepackage{amsmath,amssymb,amstext}
\usepackage{babel}



\usepackage{times}
\expandafter\ifx\csname T@LGR\endcsname\relax
\else
% LGR was declared as font encoding
  \substitutefont{LGR}{\rmdefault}{cmr}
  \substitutefont{LGR}{\sfdefault}{cmss}
  \substitutefont{LGR}{\ttdefault}{cmtt}
\fi
\expandafter\ifx\csname T@X2\endcsname\relax
  \expandafter\ifx\csname T@T2A\endcsname\relax
  \else
  % T2A was declared as font encoding
    \substitutefont{T2A}{\rmdefault}{cmr}
    \substitutefont{T2A}{\sfdefault}{cmss}
    \substitutefont{T2A}{\ttdefault}{cmtt}
  \fi
\else
% X2 was declared as font encoding
  \substitutefont{X2}{\rmdefault}{cmr}
  \substitutefont{X2}{\sfdefault}{cmss}
  \substitutefont{X2}{\ttdefault}{cmtt}
\fi


\usepackage[Bjarne]{fncychap}
\usepackage{sphinx}

\fvset{fontsize=\small}
\usepackage{geometry}


% Include hyperref last.
\usepackage{hyperref}
% Fix anchor placement for figures with captions.
\usepackage{hypcap}% it must be loaded after hyperref.
% Set up styles of URL: it should be placed after hyperref.
\urlstyle{same}

\usepackage{sphinxmessages}
\setcounter{tocdepth}{3}
\setcounter{secnumdepth}{3}


\title{VHH Plugin Package: Shot Boundary Detection (vhh\_sbd)}
\date{Jun 08, 2020}
\release{1.0.0}
\author{Daniel Helm}
\newcommand{\sphinxlogo}{\vbox{}}
\renewcommand{\releasename}{Release}
\makeindex
\begin{document}

\pagestyle{empty}
\sphinxmaketitle
\pagestyle{plain}
\sphinxtableofcontents
\pagestyle{normal}
\phantomsection\label{\detokenize{index::doc}}


The following list give an overview of the folder structure of this python repository:

\sphinxstyleemphasis{name of repository}: vhh\_sbd
\begin{itemize}
\item {} \begin{description}
\item[{\sphinxstylestrong{ApiSphinxDocumentation/}: includes all files to generate the documentation as well as the created documentations}] \leavevmode
(html, pdf)

\end{description}

\item {} 
\sphinxstylestrong{config/}: this folder includes the required configuration file

\item {} 
\sphinxstylestrong{sbd/}: this folder represents the shot\sphinxhyphen{}type\sphinxhyphen{}classification module and builds the main part of this repository

\item {} 
\sphinxstylestrong{Demo/}: this folder includes a demo script to demonstrate how the package have to be used in customized applications

\item {} \begin{description}
\item[{\sphinxstylestrong{Develop/}: includes scripts to evaluate the implemented approach as well as several helper scripts used during}] \leavevmode
development stage. Furthermore, a script is included to create the package documentation (pdf, html).

\end{description}

\item {} 
\sphinxstylestrong{README.md}: this file gives a brief description of this repository (e.g. link to this documentation)

\item {} \begin{description}
\item[{\sphinxstylestrong{requirements.txt}: this file holds all python lib dependencies and is needed to install the package in your own}] \leavevmode
virtual environment

\end{description}

\item {} 
\sphinxstylestrong{setup.py}: this script is needed to install the sbd package in your own virtual environment

\end{itemize}


\chapter{Setup  instructions}
\label{\detokenize{index:setup-instructions}}
This package includes a setup.py script and a requirements.txt file which are needed to install this package for custom
applications. The following instructions have to be done to used this library in your own application:

Requirements:
\begin{itemize}
\item {} 
Ubuntu 18.04 LTS

\item {} 
CUDA 10.1 + cuDNN

\item {} 
python version 3.6.x

\end{itemize}

Create a virtual environment:
\begin{itemize}
\item {} 
create a folder to a specified path (e.g. /xxx/vhh\_sbd/)

\item {} 
python3 \sphinxhyphen{}m venv /xxx/vhh\_sbd/

\end{itemize}

Activate the environment:
\begin{itemize}
\item {} 
source /xxx/vhh\_sbd/bin/activate

\end{itemize}

Checkout vhh\_sbd repository to a specified folder:
\begin{itemize}
\item {} 
git clone \sphinxurl{https://github.com/dahe-cvl/vhh\_sbd}

\end{itemize}

Install the sbd package and all dependencies:
\begin{itemize}
\item {} 
change to the root directory of the repository (includes setup.py)

\item {} 
python setup.py install

\end{itemize}

\begin{sphinxadmonition}{note}{Note:}
You can check the success of the installation by using the commend \sphinxstyleemphasis{pip list}. This command should give you a list with all installed python packages and it should include \sphinxstyleemphasis{vhh\_sbd}
\end{sphinxadmonition}

\begin{sphinxadmonition}{note}{Note:}
Currently there is an issue in the \sphinxstyleemphasis{setup.py} script. Therefore the pytorch libraries have to be installed manually by running the following command:
\sphinxstyleemphasis{pip install torch==1.5.0+cu101 torchvision==0.6.0+cu101 \sphinxhyphen{}f https://download.pytorch.org/whl/torch\_stable.html}
\end{sphinxadmonition}


\chapter{Parameter Description}
\label{\detokenize{index:parameter-description}}
DEBUG\_FLAG
This parameter is used to activate or deactivate the debug mode.



RESIZE\_DIM
This flag is used to to specify the resize dimension. (only usable if DOWNSCALE\_FLAG is active).



CONVERT2GRAY
This flag is used to convert a input frame into a grayscale frame (0… deactivate, 1 … activate).



CROP
This flag is used to center crop a input frame (0… deactivate, 1 … activate).



DOWNSCALE
This flag is used to scale a input frame into the specified dimension (0… deactivate, 1 … activate).



HISTOGRAM\_EQU
This parameter is used to to specify a valid pre\sphinxhyphen{}processing method (clahe” or “classic” or “none).



CANDIDATE\_SELECTION
This flag is used to to enable or disable the candidate selection mode.



SAVE\_RAW\_RESULTS
This parameter is used to save raw results (e.g. debug visualizations).



PATH\_RAW\_RESULTS
This parameter is used to specify the path for saving the raw results.



PREFIX\_RAW\_RESULTS
This parameter is used to specify the prefix for the results file.



POSTFIX\_RAW\_RESULTS
This parameter is used to specify the postfix for the results file.



SAVE\_FINAL\_RESULTS
This parameter is used to save final results (e.g. csv list).



PATH\_FINAL\_RESULTS
This parameter is used to specify the path for saving the final results.



PREFIX\_FINAL\_RESULTS
This parameter is used to specify the prefix for the results file.



POSTFIX\_FINAL\_RESULTS
This parameter is used to specify the postfix for the results file.



PATH\_VIDEOS
This parameter is used to specify the path to the videos.



THRESHOLD\_MODE
This parameter is used to specify the threshold mode (adaptive OR fixed).



THRESHOLD
This parameter is used to specify the threshold (only in fixed threshold mode \sphinxhyphen{} {[}0\sphinxhyphen{}1{]}).



ALPHA
This parameter is used to specify the adaptive threshold.



WINDOW\_SIZE
This parameter is used to specify the window size (frames history window \sphinxhyphen{} only for adaptive mode).



BACKBONE\_CNN
This parameter is used to specify the backbone cnn model ( vgg16 OR squeezenet).



SIMILARITY\_METRIC
This parameter is used to specify the similarity metric (cosine OR euclidean).



PATH\_PRETRAINED\_MODEL
This parameter is used to specify the path to the pre\sphinxhyphen{}trained model.



SAVE\_EVAL\_RESULTS
This parameter is used to save evaluation results (e.g. visualizations, … ).



PATH\_RAW\_RESULTS
This parameter is used the raw results path.



PATH\_EVAL\_RESULTS
This parameter is used to specify the path to store the evaluation results path.



PATH\_GT\_ANNOTATIONS
This parameter is used to groundtruth annotations used for evaluation.




\chapter{API Description}
\label{\detokenize{index:api-description}}
This section gives an overview of all classes and modules in \sphinxstyleemphasis{sbd} as well as an code description.


\section{Configuration class}
\label{\detokenize{Configuration:configuration-class}}\label{\detokenize{Configuration::doc}}\index{Configuration (class in sbd.Configuration)@\spxentry{Configuration}\spxextra{class in sbd.Configuration}}

\begin{fulllineitems}
\phantomsection\label{\detokenize{Configuration:sbd.Configuration.Configuration}}\pysiglinewithargsret{\sphinxbfcode{\sphinxupquote{class }}\sphinxcode{\sphinxupquote{sbd.Configuration.}}\sphinxbfcode{\sphinxupquote{Configuration}}}{\emph{\DUrole{n}{config\_file}\DUrole{p}{:} \DUrole{n}{str}}}{}
Bases: \sphinxcode{\sphinxupquote{object}}

This class is needed to read the configuration parameters specified in the configuration.yaml file.
The instance of the class is holding all parameters during runtime.

\begin{sphinxadmonition}{note}{Note:}
e.g. ./config/config\_vhh\_test.yaml
\begin{quote}

the yaml file is separated in multiple sections
config{[}‘Development’{]}
config{[}‘PreProcessing’{]}
config{[}‘SbdCore’{]}
config{[}‘Evaluation’{]}

whereas each section should hold related and meaningful parameters.
\end{quote}
\end{sphinxadmonition}
\index{loadConfig() (sbd.Configuration.Configuration method)@\spxentry{loadConfig()}\spxextra{sbd.Configuration.Configuration method}}

\begin{fulllineitems}
\phantomsection\label{\detokenize{Configuration:sbd.Configuration.Configuration.loadConfig}}\pysiglinewithargsret{\sphinxbfcode{\sphinxupquote{loadConfig}}}{}{}
Method to load configurables from the specified configuration file

\end{fulllineitems}


\end{fulllineitems}



\section{CandidateSelection class}
\label{\detokenize{CandidateSelection:candidateselection-class}}\label{\detokenize{CandidateSelection::doc}}\index{CandidateSelection (class in sbd.DeepSBD)@\spxentry{CandidateSelection}\spxextra{class in sbd.DeepSBD}}

\begin{fulllineitems}
\phantomsection\label{\detokenize{CandidateSelection:sbd.DeepSBD.CandidateSelection}}\pysiglinewithargsret{\sphinxbfcode{\sphinxupquote{class }}\sphinxcode{\sphinxupquote{sbd.DeepSBD.}}\sphinxbfcode{\sphinxupquote{CandidateSelection}}}{\emph{\DUrole{n}{config\_instance}\DUrole{p}{:} \DUrole{n}{{\hyperref[\detokenize{Configuration:sbd.Configuration.Configuration}]{\sphinxcrossref{sbd.Configuration.Configuration}}}}}}{}
Bases: \sphinxcode{\sphinxupquote{object}}

This class is used for sbd candidate selection. It detects frames ranges of about 16 frames which includes an
abrupt cut. The loaded model is pre\sphinxhyphen{}trained on the deepsbd dataset.
\index{run() (sbd.DeepSBD.CandidateSelection method)@\spxentry{run()}\spxextra{sbd.DeepSBD.CandidateSelection method}}

\begin{fulllineitems}
\phantomsection\label{\detokenize{CandidateSelection:sbd.DeepSBD.CandidateSelection.run}}\pysiglinewithargsret{\sphinxbfcode{\sphinxupquote{run}}}{\emph{\DUrole{n}{video\_path}}}{}
This method is used to run the candidate selection process.
\begin{quote}\begin{description}
\item[{Parameters}] \leavevmode
\sphinxstyleliteralstrong{\sphinxupquote{video\_path}} \textendash{} This parameter must hold a valid path to a video file.

\item[{Returns}] \leavevmode
This method returns a numpy array with a list of all detected frames ranges.

\end{description}\end{quote}

\end{fulllineitems}


\end{fulllineitems}



\section{Evaluation class}
\label{\detokenize{Evaluation:evaluation-class}}\label{\detokenize{Evaluation::doc}}\index{Evaluation (class in sbd.Evaluation)@\spxentry{Evaluation}\spxextra{class in sbd.Evaluation}}

\begin{fulllineitems}
\phantomsection\label{\detokenize{Evaluation:sbd.Evaluation.Evaluation}}\pysiglinewithargsret{\sphinxbfcode{\sphinxupquote{class }}\sphinxcode{\sphinxupquote{sbd.Evaluation.}}\sphinxbfcode{\sphinxupquote{Evaluation}}}{\emph{\DUrole{n}{config\_file}\DUrole{p}{:} \DUrole{n}{str}}}{}
Bases: \sphinxcode{\sphinxupquote{object}}

This class is used to evaluate the implemented algorithm.
\index{calculateEvaluationMetrics() (sbd.Evaluation.Evaluation method)@\spxentry{calculateEvaluationMetrics()}\spxextra{sbd.Evaluation.Evaluation method}}

\begin{fulllineitems}
\phantomsection\label{\detokenize{Evaluation:sbd.Evaluation.Evaluation.calculateEvaluationMetrics}}\pysiglinewithargsret{\sphinxbfcode{\sphinxupquote{calculateEvaluationMetrics}}}{}{}
This method is used to calculate the evaluation metrics.
\begin{quote}\begin{description}
\item[{Returns}] \leavevmode
This methods returns a numpy array including a list of the calculated metrics (precision, recall, …).

\end{description}\end{quote}

\end{fulllineitems}

\index{calculateMetrics() (sbd.Evaluation.Evaluation method)@\spxentry{calculateMetrics()}\spxextra{sbd.Evaluation.Evaluation method}}

\begin{fulllineitems}
\phantomsection\label{\detokenize{Evaluation:sbd.Evaluation.Evaluation.calculateMetrics}}\pysiglinewithargsret{\sphinxbfcode{\sphinxupquote{calculateMetrics}}}{\emph{\DUrole{n}{tp\_cnt}}, \emph{\DUrole{n}{fp\_cnt}}, \emph{\DUrole{n}{tn\_cnt}}, \emph{\DUrole{n}{fn\_cnt}}}{}
This method is used to calculate the evaluation metrics precision, recall and f1score.
\begin{quote}\begin{description}
\item[{Parameters}] \leavevmode\begin{itemize}
\item {} 
\sphinxstyleliteralstrong{\sphinxupquote{tp\_cnt}} \textendash{} This parameter must hold a valid integer representing the tp counter.

\item {} 
\sphinxstyleliteralstrong{\sphinxupquote{fp\_cnt}} \textendash{} This parameter must hold a valid integer representing the fp counter.

\item {} 
\sphinxstyleliteralstrong{\sphinxupquote{tn\_cnt}} \textendash{} This parameter must hold a valid integer representing the tn counter.

\item {} 
\sphinxstyleliteralstrong{\sphinxupquote{fn\_cnt}} \textendash{} This parameter must hold a valid integer representing the fn counter.

\end{itemize}

\item[{Returns}] \leavevmode
This method returns the scores for precision, recall, accuracy, f1\_score, tp\_rate and fp\_rate.

\end{description}\end{quote}

\end{fulllineitems}

\index{calculateSimilarityMetric() (sbd.Evaluation.Evaluation method)@\spxentry{calculateSimilarityMetric()}\spxextra{sbd.Evaluation.Evaluation method}}

\begin{fulllineitems}
\phantomsection\label{\detokenize{Evaluation:sbd.Evaluation.Evaluation.calculateSimilarityMetric}}\pysiglinewithargsret{\sphinxbfcode{\sphinxupquote{calculateSimilarityMetric}}}{\emph{\DUrole{n}{results\_np}\DUrole{p}{:} \DUrole{n}{numpy.ndarray}}, \emph{\DUrole{n}{threshold}\DUrole{o}{=}\DUrole{default_value}{4.5}}}{}
This method is used to calculate the similarity metrics based on the pre\sphinxhyphen{}calculated raw results.
\begin{quote}\begin{description}
\item[{Parameters}] \leavevmode\begin{itemize}
\item {} 
\sphinxstyleliteralstrong{\sphinxupquote{results\_np}} \textendash{} This parameter must hold a valid numpy array.

\item {} 
\sphinxstyleliteralstrong{\sphinxupquote{threshold}} \textendash{} This parameter holds a threshold. (default: 4.5)

\end{itemize}

\item[{Returns}] \leavevmode
This method returns a numpy array including the final shot boundaries.

\end{description}\end{quote}

\end{fulllineitems}

\index{evaluation() (sbd.Evaluation.Evaluation method)@\spxentry{evaluation()}\spxextra{sbd.Evaluation.Evaluation method}}

\begin{fulllineitems}
\phantomsection\label{\detokenize{Evaluation:sbd.Evaluation.Evaluation.evaluation}}\pysiglinewithargsret{\sphinxbfcode{\sphinxupquote{evaluation}}}{\emph{\DUrole{n}{result\_np}}, \emph{\DUrole{n}{vid\_name}}}{}
This method is needed to evaluate the gathered results for a specified video.
\begin{quote}\begin{description}
\item[{Parameters}] \leavevmode\begin{itemize}
\item {} 
\sphinxstyleliteralstrong{\sphinxupquote{result\_np}} \textendash{} This parameter must hold a valid numpy array.

\item {} 
\sphinxstyleliteralstrong{\sphinxupquote{vid\_name}} \textendash{} This parameter represents a video name.

\end{itemize}

\item[{Returns}] \leavevmode
This method returns the calculated TP, TN, FP and FN counters.

\end{description}\end{quote}

\end{fulllineitems}

\index{export2CSV() (sbd.Evaluation.Evaluation method)@\spxentry{export2CSV()}\spxextra{sbd.Evaluation.Evaluation method}}

\begin{fulllineitems}
\phantomsection\label{\detokenize{Evaluation:sbd.Evaluation.Evaluation.export2CSV}}\pysiglinewithargsret{\sphinxbfcode{\sphinxupquote{export2CSV}}}{\emph{\DUrole{n}{data\_np}\DUrole{p}{:} \DUrole{n}{numpy.ndarray}}, \emph{\DUrole{n}{header}\DUrole{p}{:} \DUrole{n}{str}}, \emph{\DUrole{n}{filename}\DUrole{p}{:} \DUrole{n}{str}}, \emph{\DUrole{n}{path}\DUrole{p}{:} \DUrole{n}{str}}}{}
This method is used to export the gathered results to a csv file.
\begin{quote}\begin{description}
\item[{Parameters}] \leavevmode\begin{itemize}
\item {} 
\sphinxstyleliteralstrong{\sphinxupquote{data\_np}} \textendash{} This parameter holds a valid numpy array.

\item {} 
\sphinxstyleliteralstrong{\sphinxupquote{header}} \textendash{} This parameter holds a csv header line (first line in the file \sphinxhyphen{} semicolon seperated).

\item {} 
\sphinxstyleliteralstrong{\sphinxupquote{filename}} \textendash{} This parameter must hold a valid file name.

\item {} 
\sphinxstyleliteralstrong{\sphinxupquote{path}} \textendash{} THis parameter must hold a valid file path.

\end{itemize}

\end{description}\end{quote}

\end{fulllineitems}

\index{exportMovieResultsToCSV() (sbd.Evaluation.Evaluation method)@\spxentry{exportMovieResultsToCSV()}\spxextra{sbd.Evaluation.Evaluation method}}

\begin{fulllineitems}
\phantomsection\label{\detokenize{Evaluation:sbd.Evaluation.Evaluation.exportMovieResultsToCSV}}\pysiglinewithargsret{\sphinxbfcode{\sphinxupquote{exportMovieResultsToCSV}}}{\emph{\DUrole{n}{fName}}, \emph{\DUrole{n}{res\_np}}}{}
This method is used to export video results to csv file.
\begin{quote}\begin{description}
\item[{Parameters}] \leavevmode\begin{itemize}
\item {} 
\sphinxstyleliteralstrong{\sphinxupquote{filepath}} \textendash{} This parameter must hold a valid file\_path.

\item {} 
\sphinxstyleliteralstrong{\sphinxupquote{res\_np}} \textendash{} This parameter must hold a valid numpy array containing the final results.

\end{itemize}

\end{description}\end{quote}

\end{fulllineitems}

\index{loadRawResultsFromCsv() (sbd.Evaluation.Evaluation method)@\spxentry{loadRawResultsFromCsv()}\spxextra{sbd.Evaluation.Evaluation method}}

\begin{fulllineitems}
\phantomsection\label{\detokenize{Evaluation:sbd.Evaluation.Evaluation.loadRawResultsFromCsv}}\pysiglinewithargsret{\sphinxbfcode{\sphinxupquote{loadRawResultsFromCsv}}}{\emph{\DUrole{n}{filepath}}}{}
This method is used to load raw results from csv file.
\begin{quote}\begin{description}
\item[{Parameters}] \leavevmode
\sphinxstyleliteralstrong{\sphinxupquote{filepath}} \textendash{} This parameter must hold a valid file\_path.

\item[{Returns}] \leavevmode
This method returns a numpy array containing the raw\_results.

\end{description}\end{quote}

\end{fulllineitems}

\index{loadRawResultsFromNumpy() (sbd.Evaluation.Evaluation method)@\spxentry{loadRawResultsFromNumpy()}\spxextra{sbd.Evaluation.Evaluation method}}

\begin{fulllineitems}
\phantomsection\label{\detokenize{Evaluation:sbd.Evaluation.Evaluation.loadRawResultsFromNumpy}}\pysiglinewithargsret{\sphinxbfcode{\sphinxupquote{loadRawResultsFromNumpy}}}{\emph{\DUrole{n}{filepath}}}{}
This method is used to load raw results from numpy array.
\begin{quote}\begin{description}
\item[{Parameters}] \leavevmode
\sphinxstyleliteralstrong{\sphinxupquote{filepath}} \textendash{} This parameter must hold a valid file\_path.

\item[{Returns}] \leavevmode
This method returns a numpy array containing the raw\_results.

\end{description}\end{quote}

\end{fulllineitems}

\index{loadResultsFromCSV() (sbd.Evaluation.Evaluation method)@\spxentry{loadResultsFromCSV()}\spxextra{sbd.Evaluation.Evaluation method}}

\begin{fulllineitems}
\phantomsection\label{\detokenize{Evaluation:sbd.Evaluation.Evaluation.loadResultsFromCSV}}\pysiglinewithargsret{\sphinxbfcode{\sphinxupquote{loadResultsFromCSV}}}{\emph{\DUrole{n}{filepath}}}{}
This method is used to load final results from csv file.
\begin{quote}\begin{description}
\item[{Parameters}] \leavevmode
\sphinxstyleliteralstrong{\sphinxupquote{filepath}} \textendash{} This parameter must hold a valid file\_path.

\item[{Returns}] \leavevmode
This method returns a numpy array containing the final results.

\end{description}\end{quote}

\end{fulllineitems}

\index{plotPRCurve() (sbd.Evaluation.Evaluation method)@\spxentry{plotPRCurve()}\spxextra{sbd.Evaluation.Evaluation method}}

\begin{fulllineitems}
\phantomsection\label{\detokenize{Evaluation:sbd.Evaluation.Evaluation.plotPRCurve}}\pysiglinewithargsret{\sphinxbfcode{\sphinxupquote{plotPRCurve}}}{\emph{\DUrole{n}{results\_np}}}{}
This method is needed to create and plot the precision\_recall curve.
\begin{quote}\begin{description}
\item[{Parameters}] \leavevmode
\sphinxstyleliteralstrong{\sphinxupquote{results\_np}} \textendash{} This parameter must hold a vaild numpy array including the precision and recall scores.

\end{description}\end{quote}

\end{fulllineitems}

\index{plotROCCurve() (sbd.Evaluation.Evaluation method)@\spxentry{plotROCCurve()}\spxextra{sbd.Evaluation.Evaluation method}}

\begin{fulllineitems}
\phantomsection\label{\detokenize{Evaluation:sbd.Evaluation.Evaluation.plotROCCurve}}\pysiglinewithargsret{\sphinxbfcode{\sphinxupquote{plotROCCurve}}}{\emph{\DUrole{n}{results\_np}}}{}
This method is needed to create and plot the roc curve.
\begin{quote}\begin{description}
\item[{Parameters}] \leavevmode
\sphinxstyleliteralstrong{\sphinxupquote{results\_np}} \textendash{} This parameter must hold a vaild numpy array including the precision and recall scores.

\end{description}\end{quote}

\end{fulllineitems}

\index{run() (sbd.Evaluation.Evaluation method)@\spxentry{run()}\spxextra{sbd.Evaluation.Evaluation method}}

\begin{fulllineitems}
\phantomsection\label{\detokenize{Evaluation:sbd.Evaluation.Evaluation.run}}\pysiglinewithargsret{\sphinxbfcode{\sphinxupquote{run}}}{}{}
This method is needed to run the evaluation process.

\end{fulllineitems}


\end{fulllineitems}



\section{PreProcessing class}
\label{\detokenize{PreProcessing:preprocessing-class}}\label{\detokenize{PreProcessing::doc}}\index{PreProcessing (class in sbd.PreProcessing)@\spxentry{PreProcessing}\spxextra{class in sbd.PreProcessing}}

\begin{fulllineitems}
\phantomsection\label{\detokenize{PreProcessing:sbd.PreProcessing.PreProcessing}}\pysiglinewithargsret{\sphinxbfcode{\sphinxupquote{class }}\sphinxcode{\sphinxupquote{sbd.PreProcessing.}}\sphinxbfcode{\sphinxupquote{PreProcessing}}}{\emph{\DUrole{n}{config\_instance}\DUrole{p}{:} \DUrole{n}{{\hyperref[\detokenize{Configuration:sbd.Configuration.Configuration}]{\sphinxcrossref{sbd.Configuration.Configuration}}}}}}{}
Bases: \sphinxcode{\sphinxupquote{object}}

This class is used to pre\sphinxhyphen{}process frames.
\index{applyTransformOnImg() (sbd.PreProcessing.PreProcessing method)@\spxentry{applyTransformOnImg()}\spxextra{sbd.PreProcessing.PreProcessing method}}

\begin{fulllineitems}
\phantomsection\label{\detokenize{PreProcessing:sbd.PreProcessing.PreProcessing.applyTransformOnImg}}\pysiglinewithargsret{\sphinxbfcode{\sphinxupquote{applyTransformOnImg}}}{\emph{\DUrole{n}{image}\DUrole{p}{:} \DUrole{n}{numpy.ndarray}}}{{ $\rightarrow$ numpy.ndarray}}
This method is used to apply the configured pre\sphinxhyphen{}processing methods on a numpy frame.
\begin{quote}\begin{description}
\item[{Parameters}] \leavevmode
\sphinxstyleliteralstrong{\sphinxupquote{image}} \textendash{} This parameter must hold a valid numpy image (WxHxC).

\item[{Returns}] \leavevmode
This methods returns the preprocessed numpy image.

\end{description}\end{quote}

\end{fulllineitems}

\index{applyTransformOnImgSeq() (sbd.PreProcessing.PreProcessing method)@\spxentry{applyTransformOnImgSeq()}\spxextra{sbd.PreProcessing.PreProcessing method}}

\begin{fulllineitems}
\phantomsection\label{\detokenize{PreProcessing:sbd.PreProcessing.PreProcessing.applyTransformOnImgSeq}}\pysiglinewithargsret{\sphinxbfcode{\sphinxupquote{applyTransformOnImgSeq}}}{\emph{\DUrole{n}{img\_seq}\DUrole{p}{:} \DUrole{n}{numpy.ndarray}}}{{ $\rightarrow$ numpy.ndarray}}
\end{fulllineitems}

\index{claHE() (sbd.PreProcessing.PreProcessing method)@\spxentry{claHE()}\spxextra{sbd.PreProcessing.PreProcessing method}}

\begin{fulllineitems}
\phantomsection\label{\detokenize{PreProcessing:sbd.PreProcessing.PreProcessing.claHE}}\pysiglinewithargsret{\sphinxbfcode{\sphinxupquote{claHE}}}{\emph{\DUrole{n}{img}\DUrole{p}{:} \DUrole{n}{numpy.ndarray}}}{}
This method is used to calculate the Contrast Limited Adaptive Histogram Equalization.
\begin{quote}\begin{description}
\item[{Parameters}] \leavevmode
\sphinxstyleliteralstrong{\sphinxupquote{img}} \textendash{} This parameter must hold a valid numpy image.

\item[{Returns}] \leavevmode
This method returns the pre\sphinxhyphen{}processed image.

\end{description}\end{quote}

\end{fulllineitems}

\index{classicHE() (sbd.PreProcessing.PreProcessing method)@\spxentry{classicHE()}\spxextra{sbd.PreProcessing.PreProcessing method}}

\begin{fulllineitems}
\phantomsection\label{\detokenize{PreProcessing:sbd.PreProcessing.PreProcessing.classicHE}}\pysiglinewithargsret{\sphinxbfcode{\sphinxupquote{classicHE}}}{\emph{\DUrole{n}{img}\DUrole{p}{:} \DUrole{n}{numpy.ndarray}}}{}
This method is used to calculate the classic histogram equalization.
\begin{quote}\begin{description}
\item[{Parameters}] \leavevmode
\sphinxstyleliteralstrong{\sphinxupquote{img}} \textendash{} This parameter must hold a valid numpy image.

\item[{Returns}] \leavevmode
This method returns the pre\sphinxhyphen{}processed image.

\end{description}\end{quote}

\end{fulllineitems}

\index{convertRGB2Gray() (sbd.PreProcessing.PreProcessing method)@\spxentry{convertRGB2Gray()}\spxextra{sbd.PreProcessing.PreProcessing method}}

\begin{fulllineitems}
\phantomsection\label{\detokenize{PreProcessing:sbd.PreProcessing.PreProcessing.convertRGB2Gray}}\pysiglinewithargsret{\sphinxbfcode{\sphinxupquote{convertRGB2Gray}}}{\emph{\DUrole{n}{img}\DUrole{p}{:} \DUrole{n}{numpy.ndarray}}}{}
This method is used to convert a RBG numpy image to a grayscale image.
\begin{quote}\begin{description}
\item[{Parameters}] \leavevmode
\sphinxstyleliteralstrong{\sphinxupquote{img}} \textendash{} This parameter must hold a valid numpy image.

\item[{Returns}] \leavevmode
This method returns a grayscale image (WxHx1).

\end{description}\end{quote}

\end{fulllineitems}

\index{crop() (sbd.PreProcessing.PreProcessing method)@\spxentry{crop()}\spxextra{sbd.PreProcessing.PreProcessing method}}

\begin{fulllineitems}
\phantomsection\label{\detokenize{PreProcessing:sbd.PreProcessing.PreProcessing.crop}}\pysiglinewithargsret{\sphinxbfcode{\sphinxupquote{crop}}}{\emph{\DUrole{n}{img}\DUrole{p}{:} \DUrole{n}{numpy.ndarray}}, \emph{\DUrole{n}{dim}\DUrole{p}{:} \DUrole{n}{tuple}}}{}
This method is used to crop a specified region of interest from a given image.
\begin{quote}\begin{description}
\item[{Parameters}] \leavevmode\begin{itemize}
\item {} 
\sphinxstyleliteralstrong{\sphinxupquote{img}} \textendash{} This parameter must hold a valid numpy image.

\item {} 
\sphinxstyleliteralstrong{\sphinxupquote{dim}} \textendash{} This parameter must hold a valid tuple including the crop dimensions.

\end{itemize}

\item[{Returns}] \leavevmode
This method returns the cropped image.

\end{description}\end{quote}

\end{fulllineitems}

\index{resize() (sbd.PreProcessing.PreProcessing method)@\spxentry{resize()}\spxextra{sbd.PreProcessing.PreProcessing method}}

\begin{fulllineitems}
\phantomsection\label{\detokenize{PreProcessing:sbd.PreProcessing.PreProcessing.resize}}\pysiglinewithargsret{\sphinxbfcode{\sphinxupquote{resize}}}{\emph{\DUrole{n}{img}\DUrole{p}{:} \DUrole{n}{numpy.ndarray}}, \emph{\DUrole{n}{dim}\DUrole{p}{:} \DUrole{n}{tuple}}}{}
This method is used to resize a image.
\begin{quote}\begin{description}
\item[{Parameters}] \leavevmode\begin{itemize}
\item {} 
\sphinxstyleliteralstrong{\sphinxupquote{img}} \textendash{} This parameter must hold a valid numpy image.

\item {} 
\sphinxstyleliteralstrong{\sphinxupquote{dim}} \textendash{} This parameter must hold a valid tuple including the resize dimensions.

\end{itemize}

\item[{Returns}] \leavevmode
This method returns the resized image.

\end{description}\end{quote}

\end{fulllineitems}


\end{fulllineitems}



\section{SBD class}
\label{\detokenize{SBD:sbd-class}}\label{\detokenize{SBD::doc}}\index{SBD (class in sbd.SBD)@\spxentry{SBD}\spxextra{class in sbd.SBD}}

\begin{fulllineitems}
\phantomsection\label{\detokenize{SBD:sbd.SBD.SBD}}\pysiglinewithargsret{\sphinxbfcode{\sphinxupquote{class }}\sphinxcode{\sphinxupquote{sbd.SBD.}}\sphinxbfcode{\sphinxupquote{SBD}}}{\emph{\DUrole{n}{config\_file}\DUrole{p}{:} \DUrole{n}{str}}}{}
Bases: \sphinxcode{\sphinxupquote{object}}

Main class of shot boundary detection (sbd) package.
\index{calculateDistance() (sbd.SBD.SBD method)@\spxentry{calculateDistance()}\spxextra{sbd.SBD.SBD method}}

\begin{fulllineitems}
\phantomsection\label{\detokenize{SBD:sbd.SBD.SBD.calculateDistance}}\pysiglinewithargsret{\sphinxbfcode{\sphinxupquote{calculateDistance}}}{\emph{\DUrole{n}{x}}, \emph{\DUrole{n}{y}}}{}
This method is used to calculate the distance between 2 feature vectors.
\begin{quote}\begin{description}
\item[{Parameters}] \leavevmode\begin{itemize}
\item {} 
\sphinxstyleliteralstrong{\sphinxupquote{x}} \textendash{} This parameter represents a feature vector (one\sphinxhyphen{}dimensional)

\item {} 
\sphinxstyleliteralstrong{\sphinxupquote{y}} \textendash{} This parameter represents a feature vector (one\sphinxhyphen{}dimensional)

\end{itemize}

\item[{Returns}] \leavevmode
This method returns the similarity score of a specified distance metric.

\end{description}\end{quote}

\end{fulllineitems}

\index{convertShotBoundaries2Shots() (sbd.SBD.SBD method)@\spxentry{convertShotBoundaries2Shots()}\spxextra{sbd.SBD.SBD method}}

\begin{fulllineitems}
\phantomsection\label{\detokenize{SBD:sbd.SBD.SBD.convertShotBoundaries2Shots}}\pysiglinewithargsret{\sphinxbfcode{\sphinxupquote{convertShotBoundaries2Shots}}}{\emph{\DUrole{n}{shot\_boundaries\_np}\DUrole{p}{:} \DUrole{n}{numpy.ndarray}}}{}
This method converts a list with detected shot boundaries to the final shots.
\begin{quote}\begin{description}
\item[{Parameters}] \leavevmode
\sphinxstyleliteralstrong{\sphinxupquote{shot\_boundaries\_np}} \textendash{} This parameter must hold a numpy array with all detected shot boundaries.

\item[{Returns}] \leavevmode
This method returns a numpy list with the final shots.

\end{description}\end{quote}

\end{fulllineitems}

\index{exportFinalResultsToCsv() (sbd.SBD.SBD method)@\spxentry{exportFinalResultsToCsv()}\spxextra{sbd.SBD.SBD method}}

\begin{fulllineitems}
\phantomsection\label{\detokenize{SBD:sbd.SBD.SBD.exportFinalResultsToCsv}}\pysiglinewithargsret{\sphinxbfcode{\sphinxupquote{exportFinalResultsToCsv}}}{\emph{\DUrole{n}{shot\_l}\DUrole{p}{:} \DUrole{n}{list}}, \emph{\DUrole{n}{name}\DUrole{p}{:} \DUrole{n}{str}}}{}
This method is used to export the final results to a csv file (semicolon seperated).
:param shot\_l: This parameter must hold a valid array list including the final results list.
:param name: This parameter represents the name of the csv list.

\end{fulllineitems}

\index{exportRawResultsAsCsv\_New() (sbd.SBD.SBD method)@\spxentry{exportRawResultsAsCsv\_New()}\spxextra{sbd.SBD.SBD method}}

\begin{fulllineitems}
\phantomsection\label{\detokenize{SBD:sbd.SBD.SBD.exportRawResultsAsCsv_New}}\pysiglinewithargsret{\sphinxbfcode{\sphinxupquote{exportRawResultsAsCsv\_New}}}{\emph{\DUrole{n}{results\_np}\DUrole{p}{:} \DUrole{n}{numpy.ndarray}}}{}
This method is used to export the raw results to a csv file.
\begin{quote}\begin{description}
\item[{Parameters}] \leavevmode
\sphinxstyleliteralstrong{\sphinxupquote{results\_np}} \textendash{} This parameter must hold a valid numpy list including the raw results.

\end{description}\end{quote}

\end{fulllineitems}

\index{exportRawResultsAsNumpy() (sbd.SBD.SBD method)@\spxentry{exportRawResultsAsNumpy()}\spxextra{sbd.SBD.SBD method}}

\begin{fulllineitems}
\phantomsection\label{\detokenize{SBD:sbd.SBD.SBD.exportRawResultsAsNumpy}}\pysiglinewithargsret{\sphinxbfcode{\sphinxupquote{exportRawResultsAsNumpy}}}{\emph{\DUrole{n}{results\_np}\DUrole{p}{:} \DUrole{n}{numpy.ndarray}}}{}
This method is used to export the raw results to a numpy file.
\begin{quote}\begin{description}
\item[{Parameters}] \leavevmode
\sphinxstyleliteralstrong{\sphinxupquote{results\_np}} \textendash{} This parameter must hold a valid numpy list including the raw results.

\end{description}\end{quote}

\end{fulllineitems}

\index{runOnFolder() (sbd.SBD.SBD method)@\spxentry{runOnFolder()}\spxextra{sbd.SBD.SBD method}}

\begin{fulllineitems}
\phantomsection\label{\detokenize{SBD:sbd.SBD.SBD.runOnFolder}}\pysiglinewithargsret{\sphinxbfcode{\sphinxupquote{runOnFolder}}}{}{}
This method is used to run sbd on all video files included in a specified folder.
\begin{quote}\begin{description}
\item[{Returns}] \leavevmode
This method returns a numpy list of all detected shots in all videos.

\end{description}\end{quote}

\end{fulllineitems}

\index{runOnSingleVideo() (sbd.SBD.SBD method)@\spxentry{runOnSingleVideo()}\spxextra{sbd.SBD.SBD method}}

\begin{fulllineitems}
\phantomsection\label{\detokenize{SBD:sbd.SBD.SBD.runOnSingleVideo}}\pysiglinewithargsret{\sphinxbfcode{\sphinxupquote{runOnSingleVideo}}}{\emph{\DUrole{n}{video\_filename}}, \emph{\DUrole{n}{max\_recall\_id}\DUrole{o}{=}\DUrole{default_value}{\sphinxhyphen{} 1}}}{}
Method to run sbd on specified video.
\begin{quote}\begin{description}
\item[{Parameters}] \leavevmode\begin{itemize}
\item {} 
\sphinxstyleliteralstrong{\sphinxupquote{video\_filename}} \textendash{} This parameter must hold a valid video file path.

\item {} 
\sphinxstyleliteralstrong{\sphinxupquote{max\_recall\_id}} \textendash{} {[}required{]} integer value holding unique video id from VHH MMSI system

\end{itemize}

\end{description}\end{quote}

\end{fulllineitems}

\index{runWithCandidateSelection() (sbd.SBD.SBD method)@\spxentry{runWithCandidateSelection()}\spxextra{sbd.SBD.SBD method}}

\begin{fulllineitems}
\phantomsection\label{\detokenize{SBD:sbd.SBD.SBD.runWithCandidateSelection}}\pysiglinewithargsret{\sphinxbfcode{\sphinxupquote{runWithCandidateSelection}}}{\emph{\DUrole{n}{candidates\_np}}}{}
This method is used to run sbd with candidate selection mode.
\begin{quote}\begin{description}
\item[{Parameters}] \leavevmode
\sphinxstyleliteralstrong{\sphinxupquote{candidates\_np}} \textendash{} THis parameter must hold a valid numpy list including all pre\sphinxhyphen{}selected candidates.

\item[{Returns}] \leavevmode
This method returns a numpy list with all detected shots in a video.

\end{description}\end{quote}

\end{fulllineitems}

\index{runWithoutCandidateSelection() (sbd.SBD.SBD method)@\spxentry{runWithoutCandidateSelection()}\spxextra{sbd.SBD.SBD method}}

\begin{fulllineitems}
\phantomsection\label{\detokenize{SBD:sbd.SBD.SBD.runWithoutCandidateSelection}}\pysiglinewithargsret{\sphinxbfcode{\sphinxupquote{runWithoutCandidateSelection}}}{\emph{\DUrole{n}{src\_path}}, \emph{\DUrole{n}{vid\_name}}}{}
This method is used to run sbd without candidate selection mode.
\begin{quote}\begin{description}
\item[{Parameters}] \leavevmode\begin{itemize}
\item {} 
\sphinxstyleliteralstrong{\sphinxupquote{src\_path}} \textendash{} THis parameter must hold a valid path to the video file.

\item {} 
\sphinxstyleliteralstrong{\sphinxupquote{vid\_name}} \textendash{} This parameter must hold a valid videofile name.

\end{itemize}

\item[{Returns}] \leavevmode
This method returns a numpy list with all detected shots in a video.

\end{description}\end{quote}

\end{fulllineitems}


\end{fulllineitems}



\section{Shot class}
\label{\detokenize{Shot:shot-class}}\label{\detokenize{Shot::doc}}\index{Shot (class in sbd.Shot)@\spxentry{Shot}\spxextra{class in sbd.Shot}}

\begin{fulllineitems}
\phantomsection\label{\detokenize{Shot:sbd.Shot.Shot}}\pysiglinewithargsret{\sphinxbfcode{\sphinxupquote{class }}\sphinxcode{\sphinxupquote{sbd.Shot.}}\sphinxbfcode{\sphinxupquote{Shot}}}{\emph{\DUrole{n}{sid}}, \emph{\DUrole{n}{movie\_name}}, \emph{\DUrole{n}{start\_pos}}, \emph{\DUrole{n}{end\_pos}}}{}
Bases: \sphinxcode{\sphinxupquote{object}}

This class represents on shot and contains shot properties such as start/end frame index of a shot, shot\sphinxhyphen{}id and
video\_name.
\index{convert2String() (sbd.Shot.Shot method)@\spxentry{convert2String()}\spxextra{sbd.Shot.Shot method}}

\begin{fulllineitems}
\phantomsection\label{\detokenize{Shot:sbd.Shot.Shot.convert2String}}\pysiglinewithargsret{\sphinxbfcode{\sphinxupquote{convert2String}}}{}{}
This method is used to convert all properties of a shot into a semicolon\sphinxhyphen{}separated string.
:return:

\end{fulllineitems}

\index{printShotInfo() (sbd.Shot.Shot method)@\spxentry{printShotInfo()}\spxextra{sbd.Shot.Shot method}}

\begin{fulllineitems}
\phantomsection\label{\detokenize{Shot:sbd.Shot.Shot.printShotInfo}}\pysiglinewithargsret{\sphinxbfcode{\sphinxupquote{printShotInfo}}}{}{}
This method is used to print all properties of a shot.

\end{fulllineitems}


\end{fulllineitems}



\section{Video class}
\label{\detokenize{Video:video-class}}\label{\detokenize{Video::doc}}\index{Video (class in sbd.Video)@\spxentry{Video}\spxextra{class in sbd.Video}}

\begin{fulllineitems}
\phantomsection\label{\detokenize{Video:sbd.Video.Video}}\pysigline{\sphinxbfcode{\sphinxupquote{class }}\sphinxcode{\sphinxupquote{sbd.Video.}}\sphinxbfcode{\sphinxupquote{Video}}}
Bases: \sphinxcode{\sphinxupquote{object}}

This class represents on video and contains properties such as dimensions, length, format, video name … .
\index{getFrame() (sbd.Video.Video method)@\spxentry{getFrame()}\spxextra{sbd.Video.Video method}}

\begin{fulllineitems}
\phantomsection\label{\detokenize{Video:sbd.Video.Video.getFrame}}\pysiglinewithargsret{\sphinxbfcode{\sphinxupquote{getFrame}}}{\emph{\DUrole{n}{frame\_id}\DUrole{p}{:} \DUrole{n}{int}}}{{ $\rightarrow$ numpy.ndarray}}
This method is used to return one frame specified with a given frame index.
\begin{quote}\begin{description}
\item[{Parameters}] \leavevmode
\sphinxstyleliteralstrong{\sphinxupquote{frame\_id}} \textendash{} This parameter must hold a valid integer frame index.

\item[{Returns}] \leavevmode
This method returns a numpy frame with a specified index (position).

\end{description}\end{quote}

\end{fulllineitems}

\index{load() (sbd.Video.Video method)@\spxentry{load()}\spxextra{sbd.Video.Video method}}

\begin{fulllineitems}
\phantomsection\label{\detokenize{Video:sbd.Video.Video.load}}\pysiglinewithargsret{\sphinxbfcode{\sphinxupquote{load}}}{\emph{\DUrole{n}{vidFile}\DUrole{p}{:} \DUrole{n}{str}}}{}
This method is used to load a video of a specified storage.
\begin{quote}\begin{description}
\item[{Parameters}] \leavevmode
\sphinxstyleliteralstrong{\sphinxupquote{vidFile}} \textendash{} This parameter must hold a valid video file path.

\end{description}\end{quote}

\end{fulllineitems}

\index{printVIDInfo() (sbd.Video.Video method)@\spxentry{printVIDInfo()}\spxextra{sbd.Video.Video method}}

\begin{fulllineitems}
\phantomsection\label{\detokenize{Video:sbd.Video.Video.printVIDInfo}}\pysiglinewithargsret{\sphinxbfcode{\sphinxupquote{printVIDInfo}}}{}{}
This method is used to print all video properties.

\end{fulllineitems}


\end{fulllineitems}



\section{Utils module}
\label{\detokenize{Utils:module-sbd.utils}}\label{\detokenize{Utils:utils-module}}\label{\detokenize{Utils::doc}}\index{module@\spxentry{module}!sbd.utils@\spxentry{sbd.utils}}\index{sbd.utils@\spxentry{sbd.utils}!module@\spxentry{module}}\index{STDOUT\_TYPE (class in sbd.utils)@\spxentry{STDOUT\_TYPE}\spxextra{class in sbd.utils}}

\begin{fulllineitems}
\phantomsection\label{\detokenize{Utils:sbd.utils.STDOUT_TYPE}}\pysigline{\sphinxbfcode{\sphinxupquote{class }}\sphinxcode{\sphinxupquote{sbd.utils.}}\sphinxbfcode{\sphinxupquote{STDOUT\_TYPE}}}
Bases: \sphinxcode{\sphinxupquote{object}}

This class represents message types.
\index{ERROR (sbd.utils.STDOUT\_TYPE attribute)@\spxentry{ERROR}\spxextra{sbd.utils.STDOUT\_TYPE attribute}}

\begin{fulllineitems}
\phantomsection\label{\detokenize{Utils:sbd.utils.STDOUT_TYPE.ERROR}}\pysigline{\sphinxbfcode{\sphinxupquote{ERROR}}\sphinxbfcode{\sphinxupquote{ = 2}}}
\end{fulllineitems}

\index{INFO (sbd.utils.STDOUT\_TYPE attribute)@\spxentry{INFO}\spxextra{sbd.utils.STDOUT\_TYPE attribute}}

\begin{fulllineitems}
\phantomsection\label{\detokenize{Utils:sbd.utils.STDOUT_TYPE.INFO}}\pysigline{\sphinxbfcode{\sphinxupquote{INFO}}\sphinxbfcode{\sphinxupquote{ = 1}}}
\end{fulllineitems}


\end{fulllineitems}

\index{getCommandLineParams() (in module sbd.utils)@\spxentry{getCommandLineParams()}\spxextra{in module sbd.utils}}

\begin{fulllineitems}
\phantomsection\label{\detokenize{Utils:sbd.utils.getCommandLineParams}}\pysiglinewithargsret{\sphinxcode{\sphinxupquote{sbd.utils.}}\sphinxbfcode{\sphinxupquote{getCommandLineParams}}}{}{}
This function is used to read commandline parameters (e.g. just used in development stage)
:return: list of parameters.

\end{fulllineitems}

\index{printCustom() (in module sbd.utils)@\spxentry{printCustom()}\spxextra{in module sbd.utils}}

\begin{fulllineitems}
\phantomsection\label{\detokenize{Utils:sbd.utils.printCustom}}\pysiglinewithargsret{\sphinxcode{\sphinxupquote{sbd.utils.}}\sphinxbfcode{\sphinxupquote{printCustom}}}{\emph{\DUrole{n}{msg}\DUrole{p}{:} \DUrole{n}{str}}, \emph{\DUrole{n}{type}\DUrole{p}{:} \DUrole{n}{int}}}{}
This function represents a customized print function (error/info msg).
:param msg: Message to print.
:param type: Type of message (info or error).

\end{fulllineitems}



\chapter{Indices and tables}
\label{\detokenize{index:indices-and-tables}}\begin{itemize}
\item {} 
\DUrole{xref,std,std-ref}{genindex}

\item {} 
\DUrole{xref,std,std-ref}{modindex}

\item {} 
\DUrole{xref,std,std-ref}{search}

\end{itemize}


\section{References}
\label{\detokenize{index:references}}

\renewcommand{\indexname}{Python Module Index}
\begin{sphinxtheindex}
\let\bigletter\sphinxstyleindexlettergroup
\bigletter{s}
\item\relax\sphinxstyleindexentry{sbd.utils}\sphinxstyleindexpageref{Utils:\detokenize{module-sbd.utils}}
\end{sphinxtheindex}

\renewcommand{\indexname}{Index}
\printindex
\end{document}